\documentclass{experiment}

%%%%%%%% DOCUMENT %%%%%%%%
\teacherName{这是教师名字}
\classNumber{M56748989}
\studentName{学生名字}
\professional{专业名字}
\experimentReport{实验一:计算机基础实验报告}
\begin{document}
	\makecover
	\section{实验目的}
		\begin{enumerate}
		\item 熟悉常用编码器,译码器的功能逻辑。
		\item 学习组合逻辑电路译码器或编码器的设计方法及应用。
		\item 熟悉Verilog代码的编写方法。
		\end{enumerate}
	\section{实验内容}
	
	设计一个8线-3线独热码编码器,在该编码器中,输入端有8个信号线,输出端有3个信号线。
	输入数据为8位二进制数,其中只有一位为1,其余位均为0,则输出数据为对应位的二进制表示。
	

	\begin{table}[htbp]
		\centering
		\caption{真值表}
		\begin{tabular}{cccccccc|ccc}
			\toprule
			\multicolumn{8}{c|}{输入in[7:0]} & \multicolumn{3}{c}{输出out[2:0]} \\
			\cmidrule{1-8}\cmidrule{9-11} in[7] & in[6] & in[5] & in[4] & in[3] & in[2] & in[1] & in[0] & out[2] & out[1] & out[0] \\
			\midrule
			1 & 0 & 0 & 0 & 0 & 0 & 0 & 0 & 1 & 1 & 1 \\
			0 & 1 & 0 & 0 & 0 & 0 & 0 & 0 & 1 & 1 & 0 \\
			0 & 0 & 1 & 0 & 0 & 0 & 0 & 0 & 1 & 0 & 1 \\
			0 & 0 & 0 & 1 & 0 & 0 & 0 & 0 & 1 & 0 & 0 \\
			0 & 0 & 0 & 0 & 1 & 0 & 0 & 0 & 0 & 1 & 1 \\
			0 & 0 & 0 & 0 & 0 & 1 & 0 & 0 & 0 & 1 & 0 \\
			0 & 0 & 0 & 0 & 0 & 0 & 1 & 0 & 0 & 0 & 1 \\
			0 & 0 & 0 & 0 & 0 & 0 & 0 & 1 & 0 & 0 & 0 \\
			\bottomrule
		\end{tabular}%
		\label{tab:addlabel}%
	\end{table}%

	8-3编码器的verilog代码实现如下:
\begin{lstlisting}[language=verilog]
	module encoder_8_3(
		input [7:0] din,
		output reg [2:0] dout
	);
	
	always @(*)
	begin
		case(din)
			8'b00000001: dout = 3'b000;
			8'b00000010: dout = 3'b001;
			8'b00000100: dout = 3'b010;
			8'b00001000: dout = 3'b011;
			8'b00010000: dout = 3'b100;
			8'b00100000: dout = 3'b101;
			8'b01000000: dout = 3'b110;
			8'b10000000: dout = 3'b111;
		default: dout = 3'b000;
		endcase
	end
	
	endmodule
\end{lstlisting}

8-3编码器的仿真文件的verilog代码实现如下:
\begin{lstlisting}[language=verilog]
	module encoder_8_3_tb;
	
	reg [7:0] din;
	wire [2:0] dout;
	
	encoder_8_3 encoder(
		.din(din),
		.dout(dout)
	);
	
	initial begin
		din = 8'b00000001;
		#10;
		if (dout !== 3'b000) $error("Test 1 failed!");
		
		din = 8'b00000010;
		#10;
		if (dout !== 3'b001) $error("Test 2 failed!");
		
		din = 8'b00000100;
		#10;
		if (dout !== 3'b010) $error("Test 3 failed!");
		
		din = 8'b00001000;
		#10;
		if (dout !== 3'b011) $error("Test 4 failed!");
		
		din = 8'b00010000;
		#10;
		if (dout !== 3'b100) $error("Test 5 failed!");
		
		din = 8'b00100000;
		#10;
		if (dout !== 3'b101) $error("Test 6 failed!");
		
		din = 8'b01000000;
		#10;
		if (dout !== 3'b110) $error("Test 7 failed!");
		
		din = 8'b10000000;
		#10;
		if (dout !== 3'b111) $error("Test 8 failed!");
		
		$display("All tests passed.");
		$finish;
	end
	
	endmodule
\end{lstlisting}

\section{操作步骤}

\begin{enumerate}
\item 根据Vivado的设计流程,建立本次实验项目的工程文件;
\item 建立本次实验内容的Verilog功能模块,修改语法错误;
\item 建立仿真文件,进行电路的功能仿真;
\item 编译综合,直到没有错误,查看综合结果(原理图及资源利用报告);
\item 建立约束文件xdc,生成bit文件;
\item 正确连接实验箱,下载bit文件,记录实验现象及结果。
\end{enumerate}

\section{实验结果}

这里可以写具体的实验结果。

% 设置参考文献
\bibliographystyle{gbt7714-2005-author-year}
\nocite{*}
\bibliography{refers}

\end{document}
