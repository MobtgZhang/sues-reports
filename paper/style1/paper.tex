\documentclass[a4paper,12pt]{paper}


%---------------------------------------------------------------------
%	页眉页脚设置
%---------------------------------------------------------------------
\fancypagestyle{plain}{
    \pagestyle{fancy}      %改变章节首页页眉
}

\pagestyle{fancy}
\lhead{\kaishu~课程报告~}
\rhead{\kaishu~xxx}
\cfoot{\thepage}
\titleformat{\chapter}{\centering\zihao{2}\heiti}{第\chinese{chapter}章}{1em}{}
\begin{document}
%---------------------------------------------------------------------
%	封面设置
%---------------------------------------------------------------------
\begin{titlepage}
    \begin{center}
        
    %\includegraphics[width=0.60\textwidth]{logo-sues.jpg}\\
    \vspace{10mm}
    \textbf{\zihao{1}{\heiti{上海工程技术大学XX学院}}}\\[0.8cm]
    \textbf{\zihao{1}{\heiti{《XX学》课程报告}}}\\[3cm]
    \includegraphics[width=0.3\textwidth]{logo-sues-mark.jpg}\\%这里是你的照片
    \vspace{\fill}
    
\setlength{\extrarowheight}{3mm}
{\songti\zihao{3}	
\begin{tabular}{rl}
    
    {\makebox[4\ccwd][s]{学\qquad 号:}} & ~\kaishu\underline{XXXX} \\
    {\makebox[4\ccwd][s]{姓\qquad 名:}} & ~\kaishu\underline{XXXX} \\
    {\makebox[4\ccwd][s]{年\qquad 级:}} & ~\kaishu\underline{XXXX}\\
    {\makebox[4\ccwd][s]{专\qquad 业:}} & ~\kaishu\underline{XXXX}\\
    {\makebox[4\ccwd][s]{授课教师:}}  & ~\kaishu\underline{XXXX~教授}\\ 
    {\makebox[4\ccwd][s]{课程助教:}} & ~\kaishu\underline{XXXX~XXXX}\\
    {\makebox[4\ccwd][s]{完成日期:}}  & ~\kaishu\underline{\today}\\ 

\end{tabular}
 }\\[2cm]
%\vspace{\fill}
%\zihao{4}
%使用\LaTeX 撰写于\today
    \end{center}	
\end{titlepage}
%---------------------------------------------------------------------
%  摘要页
%---------------------------------------------------------------------
\begin{abstract}
    这里是摘要。
    \vspace*{15em}

    \textbf{关键词:}总结,理解,思考
\end{abstract}

\newpage

\begin{center}
    \textbf{Abstract}
\end{center}

This is abstract.
\vspace*{15em}

\textbf{Keywords } summary, comprehension, thinking


%---------------------------------------------------------------------
%  目录页
%---------------------------------------------------------------------
\newpage
\tableofcontents % 生成目录
\newpage

%---------------------------------------------------------------------
%  绪论
%---------------------------------------------------------------------
\chapter{课程理解}
\setcounter{page}{1}

\section{实验目的}

\begin{itemize}
    \item 熟悉、剖析、设计、实现直升机实验系统,获得对智能系统的基本结构及其各个组成单元的基本认识。
    \item 掌握状态反馈、观测器设计等现代控制理论。
    \item 学会运用MATLAB/Simulink 来搭建系统仿真,并在Simulink环境下实现实时控制。
    \item 学会将仿真结果与实验相结合,了解仿真和实际系统的区别与联系。
    \item 运用Word或\LaTeX 完成基本的科技报告撰写。
\end{itemize}

%---------------------------------------------------------------------
%  极点配置
%---------------------------------------------------------------------
\chapter{知识点总结} 

\section{空间描述与变换}

\begin{definition}[位姿]
    位姿是两坐标系间的相互关系,可以等价地用一个位置矢量和一个旋转矩阵来描述:$\left\{ B \right\} = \left\{ {{}_B^AR,{}^A{P_{BORG}}} \right\}$
\end{definition}

\begin{equation}
    F=ma
\end{equation}

%---------------------------------------------------------------------
%  分离原则
%---------------------------------------------------------------------
\chapter{总结与展望}

\section{深度学习方法在机械臂控制中的应用}

\cite{wilson2019learning}采用了sim-to-real learning的架构。

%---------------------------------------------------------------------
%  实验总结
%---------------------------------------------------------------------
%\titleformat{\chapter}{\centering\zihao{-1}\heiti}{}{1em}{}

%---------------------------------------------------------------------
%  参考文献设置
%---------------------------------------------------------------------
%\addcontentsline{toc}{chapter}{参考文献}

%\printbibliography

\titleformat{\chapter}{\centering\zihao{2}\heiti}{附录~\Alph{chapter}}{1em}{}

\begin{appendix}

\chapter{第一部分}

\begin{lstlisting}[language=python]
print('hello world') 
\end{lstlisting}

\chapter{第二部分}

% Please add the following required packages to your document preamble:
% \usepackage{booktabs}
\begin{table}[]
    \centering
    \caption{测试结果}
    \label{tab:my-table}
    \begin{tabular}{@{}cc@{}}
    \toprule
    算法 & 准确率 \\ \midrule
    I & 0.7684 \\
    II & 0.7865 \\
    III & 0.7655 \\ \bottomrule
    \end{tabular}
\end{table}

\end{appendix}

\end{document}
